\documentclass{res} 

\usepackage{pifont}
\newsectionwidth{0pt}  
\usepackage{fancyhdr}
\usepackage{multicol}
\usepackage{calc}
\usepackage{ragged2e}
\usepackage[bottom=1.25in,top=-0.2in,left=0.75in,right=0.75in]{geometry}
\renewcommand{\headrulewidth}{0pt}
\setlength{\headheight}{24pt}
\setlength{\headsep}{24pt}
\pagestyle{fancy}
\long\def\/*#1*/{}
\newlength{\remaining}
\newcommand{\titleline}[1]{%
    \setlength{\remaining}{\textwidth-\widthof{\textsc{#1}}}
    \noindent\underline{\textsc{#1}\hspace*{\remaining}}\par}
\cfoot{}

\begin{document}
	\thispagestyle{empty} % no header
	\centerline{\bf \Huge{Nithin Raghavan}}
    \vspace{1pt}
    \centerline{(678) 200-5839 | rnithin@berkeley.edu | rnithin1 (Github) | linkedin.com/in/nithinraghavan}
	\begin{resume}
	\vspace{-20pt}

        \section{\titleline{\centerline{EDUCATION}}}
		\vspace{6pt} 
		{\bf University of California, Berkeley (Class of 2021)} \hfill \emph{Aug 2017 -- present} \\
		Computer Science {\sl Bachelor of Arts}, Applied Mathematics {\sl Bachelor of Arts} \hfill (\emph{GPA: 3.67}) \\
		\vspace{-20pt}
		\begin{multicols}{2}
		\begin{itemize}
		\item CS61B: Data Structures%{\it (Intended)}
		\item CS170: Efficient Algorithms
		\item EE127: Optimization Models and Applications
		\item Math 128a: Numerical Analysis
		\item Math 126: Partial Differential Equations
		\item CS189: Introduction to Machine Learning %{\it (Intended)}
		\end{itemize}
		\end{multicols}
		\vspace{-25pt}
		\vspace{6pt}
		{\bf Georgia Institute of Technology} \hfill \emph{Aug 2015 -- May 2017} \\
		Courses taken while in high school \vspace{-10pt}
		\begin{multicols}{2}
		\begin{itemize}
		\item Math 3012: Applied Combinatorics
		\item Math 2803: Number Theory and Cryptography  	
		\end{itemize}
		\end{multicols}
		\vspace{-15pt}
		\vspace{-5pt}
        \section{\titleline{\centerline{EXPERIENCE}}}
		{\bf $\rightarrow$ Samsung Advanced Computing Lab} \hfill \emph{May 2019 -- Aug 2019} \\
        \vspace{-10pt}
        \begin{itemize}
%            \item Conducted extensive research on deep learning usecases and models as part of Samsung's GPU team
            \item Conducted extensive research on the potential routes of optimization and quantization of deep learning models such as MobileNet, R-FCN, SRCNN and ESRGAN as part of Samsung's GPU team
%            \item Wrote and implemented OpenCL and OpenGL code 
            \item Researched the graphics pipeline and became acquainted with Samsung's future compute architecture, and wrote + ran 2D register-blocked GEMM kernels with increased WPT in OpenCL on Samsung architecture
            \item Wrote and trained two neural networks; the first to perform ambient occlusion on complex OpenGL-rendered scenes, and the second to convert a flat-rendered scene to a lifelike, physically based rendered one  
        \end{itemize}
		\vspace{-10pt}
		{\bf $\rightarrow$ Mobile Sensing Lab, UC Berkeley} \hfill \emph{Oct 2018 -- Present} \\
        \vspace{-10pt}
        \begin{itemize}
            \item Currently writing code implementing a parallelized Frank-Wolfe algorithm for dynamic traffic assignment using contraction hierarchies in C++ and CUDA
            \item Helping research the impact of different optimization models of routing behaviour on the Waze traffic problem
        \end{itemize}
		\vspace{-10pt}
		{\bf $\rightarrow$ RISE Lab, UC Berkeley} \hfill \emph{Jun 2018 -- Dec 2018} \\
        \vspace{-10pt}
        \begin{itemize}
            \item Designed and implemented a data visualization tool for Jupyter Notebook for hyperparameter optimization for Cirrus, a serverless machine learning framework
            \item Helped write code to spawn AWS Lambdas that parallelize training of models like logistic regression
        \end{itemize}
%		\vspace{-10pt}
%		{\bf $\rightarrow$ IBM Almaden Research Center, Machine Learning Laboratory} \hfill \emph{Jul 2017 -- Aug 2017} \\
%        \vspace{-10pt}
%        \begin{itemize}
%        \item Trained an artifical neural network with visual question answering abilities on Stanford’s CLEVR
%dataset with 70\% overall accuracy
%        \item Implemented sequence autoencoders, CNNs and LSTMs with Tensorflow and Keras
%        \end{itemize}
		\vspace{-10pt}
        \section{\titleline{\centerline{PROJECTS}}}
        \vspace{6pt}
		{\bf $\rightarrow$ Resource-Provisioning GPU Server} \hfill \emph{Dec 2017 -- present} \\
        \vspace{-10pt}
        \begin{itemize}
            \item Developed a shell in Python automating on-demand request processing + resource provisioning in GPU cluster
            \item Collaborated on a team to create a program that utilizes Slurm for cluster management and deploys tasks in Docker containers
        \end{itemize}
        \vspace{-10pt}
		{\bf $\rightarrow$ Software Renderer} \hfill \emph{Jul 2019} \\
        \vspace{-10pt}
        \begin{itemize}
        \item Developed a software-based rasterizer and renderer with pixel and vertex shader support in C++, and increased its runtime with block-based rasterization
        \item Renders with SDL, and is capable of barycentric interpolation, backface culling and texture mapping
        \end{itemize}
		\vspace{-10pt}
		{\bf $\rightarrow$ LASSO/Wavelet Based Compressed Sensing Algorithm} \hfill \emph{Jul 2019} \\
        \vspace{-10pt}
        \begin{itemize}
        \item Lossily compresses audio/images by computing LASSO on the matrix-vector product representation of the discrete wavelet transform of the input signal
        \item Uses orthogonal Daubechies wavelets, and preprocesses data for ML training
        \end{itemize}
		\vspace{-10pt}
%		{\bf $\rightarrow$ TaxiFindMe} \hfill \emph{Apr 2018} \\
%        \vspace{-10pt}
%        \begin{itemize}
%        \item Routing web app that helps New Yorkers find the best spot to minimize taxi waiting time, taking into
%account travel time and time of day
%        \item Preprocessed 20 million entry taxi dataset with k-means machine learning algorithm; for querying, KNN is run from an input location to find nearest cluster. Frontend employs Django
%        \item Reduced query time up to 94\% from the naive implementation
%        \end{itemize}
%		\vspace{-10pt}
%		{\bf $\rightarrow$ ShirtMapper} \hfill \emph{Jan 2018} \\
%        \vspace{-10pt}
%        \begin{itemize}
%            \item App that resizes images of custom shirts and maps them onto people
%            \item Utilizes OpenCV and Scipy, and uses Haar classifiers for edge detection; frontend employs React Native
%        \end{itemize}
%		\vspace{-10pt}
        \section{\titleline{\centerline{SKILLS}}}
		\vspace{6pt}
		{\bf Awards:} Exploravision National Contest \hfill \emph{2016} \\
        \vspace{-10pt}
        \begin{itemize}
		    \item Wrote a paper proposing blockchain's potential link to autonomous vehicles, and won honorable mention.	
        \end{itemize}
        \vspace{-10pt}
        {\bf Models/Algorithms:} Regression/classification (ridge, logistic, SVM, decision trees, OLS), PCA/SVD, \\
        \vspace{-0pt}
        \hspace{100pt} ensemble learning, k-means, deep learning (CNNs, LSTMs, GANs), Frank-Wolfe \\
            {\bf Frameworks/Softwares:} Numpy, Scipy, Pytorch, OpenCV, Docker, Slurm, d3js, PyQt, OpenCL, OpenGL \\
			{\bf Languages:} Python, Java, C, C++, C\#, Bash, Latex, SQL, JavaScript, Matlab, RISC-V \\
            {\bf Operating Systems:} Unix-like systems (Linux, FreeBSD, Mac OS X), Windows \\
			{\bf Certifications:} Android Development (University of Maryland through Coursera) \\
		\vspace{-6pt}
			
	\end{resume} 
\end{document}
