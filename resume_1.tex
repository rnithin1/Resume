\documentclass{res} 

\usepackage{pifont}
\newsectionwidth{0pt}  
\usepackage{fancyhdr}
\usepackage{multicol}
\renewcommand{\headrulewidth}{0pt}
\setlength{\headheight}{24pt}
\setlength{\headsep}{24pt}
\pagestyle{fancy}
\long\def\/*#1*/{}
\cfoot{}
\topmargin=-0.6in

\begin{document}
	\thispagestyle{empty} % no header
	\centerline{\bf \Large{Nithin Raghavan}}
	\centerline{(678) 200-5839}
	\centerline{rnithin@berkeley.edu}
	\centerline{rnithin1 (Github)}
	\centerline{linkedin.com/in/nithinraghavan}
	\begin{resume}
	\vspace{-20pt}

\section{\centerline{EDUCATION}}
		\vspace{6pt} 
		{\bf University of California, Berkeley (Class of 2020)} \hfill \emph{August 2017 -- present} \\
		Computer Science {\sl Bachelor of Arts}, Applied Mathematics {\sl Bachelor of Arts} \hfill (\emph{GPA: 3.687}) \\
		\vspace{-20pt}
		\begin{multicols}{2}
		\begin{itemize}
		\item CS61B: Data Structures%{\it (Intended)}
		\item CS170: Efficient Algorithms
		\item EE127: Optimization Models and Applications
		\item Blockchain for Developers 
		\item Math 126: Partial Differential Equations
		\item CS70: Discrete Maths and Probability %{\it (Intended)}
		\end{itemize}
		\end{multicols}
		\vspace{-25pt}
		\vspace{6pt}
		{\bf Georgia Institute of Technology} \hfill \emph{August 2015 -- May 2017} \\
		Courses Taken while in High School
		\vspace{-10pt}
		\begin{multicols}{2}
		\begin{itemize}
		\item Linear Algebra
		\item Multivariable Calculus
		\item Applied Combinatorics
		\item Number Theory and Cryptography 	
		\end{itemize}
		\end{multicols}
		\vspace{-15pt}
		{\bf Awards:} Exploravision National Contest \hfill \emph{2016} \\
		Wrote a paper proposing blockchain's potential link to autonomous vehicles, and won honorable mention.	
	
	
		
		\vspace{-2pt}
		\section{\centerline{EXPERIENCE}}
		\vspace{6pt}
		{\bf RISE Lab, UC Berkeley} \hfill \emph{June 2018 -- present} \\
		{Working on data visualization for Cirrus, a serverless machine learning framework with logistic regression as a primitive. Currently utilizes virtual machines, but working towards using AWS Lambdas for tasks such as efficiently optimizing hyperparameters. }
		\vspace{6pt} \\
		{\bf IBM Almaden Research Center, Machine Learning Laboratory} \hfill \emph{July 2017 -- August 2017} \\
		{Used Tensorflow and Keras to create artifical neural networks implementing the bag-of-words representation in order to analyze visual reasoning abilities on the CLEVR dataset, which encouraged complex reasoning in response to sophisticated English questions. Included sequence autoencoders, CNNs and LSTMs. }
		\vspace{6pt} \\
		{\bf Georgia Institute of Technology School of Aerospace Engineering} \hfill \emph{September 2016 -- May 2017} \\
		{Helped research the development of high-bandwidth, high-efficiency methods of energy transfer using millimeter waves, involving proposed circuits which have the potential to increase efficiency of wireless energy transfer up to 90\%. Shadowed professors and graduate students working on wind tunnels.}
		\vspace{6pt} \\
		{\bf Georgia Institute of Technology School of Physics} \hfill \emph{May 2016 -- July 2016} \\
		{Shadowed professors and graduate students researching the potential impacts of the September 2015 LIGO sighting of gravitational waves. Worked with the Einstein toolkit to model relativistic astrophysical phenomena. }
		\vspace{-10pt}
		\section{\centerline{PROJECTS}}
		\vspace{6pt}
		{\bf Resource-Provisioning GPU Server} \hfill \emph{December 2017 -- present} \\
			Helped develop, and currently maintain, a program and Python-based shell to automate on-demand request processing and resource provisioning in a GPU + CPU cluster within the EECS department for UC Berkeley use. Uses Slurm for cluster management, and deploys tasks in Docker containers. \\
		\vspace{-6pt}
		%{\bf Using Keras on CLEVR Dataset to Test Visual Reasoning Prowess} \hfill \emph{August 2017} \\
		%	Used the Keras API to act on the CLEVR dataset to test artificial neural network visual reasoning abilities. 

		%{\bf Using the Einstein Toolkit to Model Binary Black Hole Merging} \hfill \emph{June -- July 2016} \\
		%	Used the Einstein Toolkit on Cactus open source problem solving environment to simulate binary black hole mergers.\\
		\vspace{-2pt}
			%React, Ember, Node.js, Express, Scalatra, Spring, Apache Camel, Django
		%\section{\centerline{AWARDS}}
		%\vspace{6pt}
		%{\bf University of California, Berkeley} \hfill \emph{Currently taking} \\
		%CS 61A, Physics 5A, Astron 84, Math 49 
		%\vspace{6pt} \\
		%{\bf Georgia Institute of Technology} \hfill \emph{August 2015 -- May 2017} \\
		%{Linear Algebra and Differential Equations, Multivariable Calculus, Applied Combinatorics, Number Theory} \\
		%{\bf Android Development Course} (certified) \hfill \emph{2015} \\
		%{Taken in University of Maryland through Coursera.} \\
		%{\bf Exploravision National Contest} \hfill \emph{2016} \\
		%{Wrote a paper proposing blockchain's potential link to autonomous vehicles, and won honorable mention.}
		\section{\centerline{SKILLS}}
		\vspace{6pt}
			{\bf Frameworks/Softwares:} Numpy, Scipy, Pytorch, Git, Unity3D, Docker, Slurm, Ethereum VM, ta-lib \\
			{\bf Programming Languages:} Python, Java, C, C++, C\#, CLisp, Bash, LaTeX, SQL, JavaScript, Solidity \\
			{\bf Operating Systems:} Unix-like systems (Linux, FreeBSD, Mac OS X), Windows \\
			{\bf Certifications:} Android Development (University of Maryland through Coursera) \\
			{\bf Languages:} English, Spanish, Tamil  
		\vspace{-6pt}
			
	\end{resume} 
\end{document}
