\documentclass{res} 

\usepackage{pifont}
\newsectionwidth{0pt}  
\usepackage{fancyhdr}
\usepackage{multicol}
\usepackage{textcomp}
\usepackage{calc}
\usepackage[bottom=1.25in,top=-0.2in,left=0.75in,right=0.75in]{geometry}
\renewcommand{\headrulewidth}{0pt}
\setlength{\headheight}{24pt}
\setlength{\headsep}{24pt}
\pagestyle{fancy}
\long\def\/*#1*/{}
\newlength{\remaining}
\newcommand{\titleline}[1]{%
    \setlength{\remaining}{\textwidth-\widthof{\textsc{#1}}}
    \noindent\underline{\textsc{#1}\hspace*{\remaining}}\par}
\cfoot{}

\begin{document}
	\thispagestyle{empty} % no header
	\centerline{\bf \Huge{Nithin Raghavan}}
    \vspace{1pt}
    \centerline{(678) 200-5839 | rnithin@berkeley.edu | rnithin1 (Github)  | csua.org/{\raise.17ex\hbox{$\scriptstyle\sim$}}rnithin}
	\begin{resume}
	\vspace{-20pt}

        \section{\titleline{\centerline{EDUCATION}}}
		\vspace{6pt} 
		{\bf University of California, Berkeley (2017 - 2021)} \hfill \emph{Aug 2017 -- present} \\
		Computer Science {\sl Bachelor of Arts}, Applied Mathematics {\sl Bachelor of Arts} \hfill (\emph{GPA: 3.67}) \\
		\vspace{-20pt}
		\begin{multicols}{2}
		\begin{itemize}
		\item CS61B: Data Structures%{\it (Intended)}
		\item CS170: Efficient Algorithms
		\item EE127: Optimization Models and Applications
		\item Math 128a: Numerical Analysis
		\item Math 126: Partial Differential Equations
		\item CS189: Introduction to Machine Learning %{\it (Intended)}
		\end{itemize}
		\end{multicols}
		\vspace{-25pt}
		\vspace{6pt}
		{\bf Georgia Institute of Technology} \hfill \emph{Aug 2015 -- May 2017} \\
		Courses Taken in High School
		\vspace{-10pt}
		\begin{multicols}{2}
		\begin{itemize}
		\item Math 3012: Applied Combinatorics
		\item Math 2803: Number Theory and Cryptography 	
		\end{itemize}
		\end{multicols}
		\vspace{-15pt}
		\vspace{-5pt}
        \section{\titleline{\centerline{EXPERIENCE}}}
		{\bf $\rightarrow$ Visual Computing Lab, UC Berkeley} \hfill \emph{Oct 2019 -- Present} \\
        \vspace{-10pt}
        \begin{itemize}
            \item Worked with several graduate students to submit a paper to NeurIPS on a new concept in multilayer perceptron theory
            \item Theory states that an input embedding of Fourier Features enables a low-dimensional MLP to learn high frequency functions
            \item Helped research volumetric octree compression on a voxel grid for the Neural Radiance Functions (NeRF) paper
            \item Currently researching several concepts in graphics involving radiance transfer and volumetric rendering
            \item Researched NTK theory, neural network theory, kernel regression theory, measure theory, Fourier theory, signal processing, and relevant graphics knowledge for the paper
        \end{itemize}
		\vspace{-10pt}
		{\bf $\rightarrow$ Ford Greenfield Labs} \hfill \emph{June 2020 -- August 2020} \\
        \vspace{-10pt}
        \begin{itemize}
            \item Worked on a neural network architecture to generate depth and segmentation maps from a single RGB image
            \item Reduces cost to almost zero, compared to several thousand dollars currently required to generate such real-world info
            \item Invention disclosure (that might result in a patent) submitted for consideration by Ford lawyers 
            \item Currently writing a paper to be submitted to CVPR
            \item Additionally, worked on dynamic route calculation project for Ford Electric Vehicles taking into account charging
                stations, weather, cost and distance of travel
        \end{itemize}
		\vspace{-10pt}
		{\bf $\rightarrow$ Samsung Advanced Computing Lab} \hfill \emph{May 2019 -- August 2019} \\
        \vspace{-10pt}
        \begin{itemize}
            \item Conducted extensive research on deep learning usecases and models as part of Samsung's GPU team
            \item Conducted extensive research on the potential routes of optimization and quantization of deep learning models such as MobileNet, R-FCN, SRCNN and ESRGAN as part of Samsung's GPU team
            \item Wrote and implemented OpenCL and OpenGL code 
            \item Researched the graphics pipeline and became acquainted with AMD's compute and graphics architecture, and wrote + ran 2D
                register-blocked GEMM kernels with increased WPT in OpenCL on AMD architecture
            \item Wrote and trained two neural networks; the first to perform ambient occlusion on complex OpenGL-rendered scenes, and the second to convert a flat-rendered scene to a lifelike, physically based rendered one  
        \end{itemize}
		\vspace{-10pt}
		{\bf $\rightarrow$ Mobile Sensing Lab, UC Berkeley} \hfill \emph{Oct 2018 -- Sept 2019} \\
        \vspace{-10pt}
        \begin{itemize}
            \item Wrote code implementing a parallelized Frank-Wolfe algorithm for dynamic traffic assignment in C++/CUDA using contraction
                hierarchies
            \item Helped research the impact of different optimization models of routing behaviour on the Waze problem
        \end{itemize}
		\vspace{-10pt}
		{\bf $\rightarrow$ RISE Lab, UC Berkeley} \hfill \emph{Jun 2018 -- Dec 2018} \\
        \vspace{-10pt}
        \begin{itemize}
            \item Designed and implemented a data visualization tool for Jupyter Notebook for hyperparameter opti-
mization for Cirrus, a serverless machine learning framework
            \item Helped write code involving AWS Lambdas for model primitives such as logistic regression
        \end{itemize}
		\vspace{-10pt}
		{\bf $\rightarrow$ IBM Almaden Research Center, Machine Learning Laboratory} \hfill \emph{Jul 2017 -- Aug 2017} \\
        \vspace{-10pt}
        \begin{itemize}
            \item Researched neural network architectures for the task of visual question answering on Stanford’s CLEVR dataset
            \item Included LSTM sequence autoencoders in conjunction with CNNs
        \end{itemize}
		\vspace{-10pt}
		{\bf $\rightarrow$ Georgia Tech School of Aerospace Engineering} \hfill \emph{Sept 2016 -- May 2017} \\
        \vspace{-10pt}
        \begin{itemize}
            \item Researched development of high bandwidth, high efficiency wireless energy transfer methods
            \item Proposed circuits with millimeter wave input and Fabry-Perot resonators
        \end{itemize}
		\vspace{-10pt}
		{\bf $\rightarrow$ Georgia Tech School of Physics} \hfill \emph{May 2016 -- Jul 2016} \\
        \vspace{-10pt}
        \begin{itemize}
            \item Shadowed professors and graduate students researching impacts of the September 2015 LIGO sighting of gravitational waves
            \item Was introduced to the Einstein Toolkit for the modelling of relativistic astrophysical phenomena
        \end{itemize}
		\vspace{-10pt}
        \section{\titleline{\centerline{PROJECTS}}}
        \vspace{6pt}
		{\bf $\rightarrow$ Resource-Provisioning GPU Server} \hfill \emph{Dec 2017 -- present} \\
        \vspace{-10pt}
        \begin{itemize}
            \item Developed a Python-based shell to automate on-demand request processing and resource provisioning
in a GPU + CPU cluster
            \item Collaborated on a team to create a program that utilizes Slurm for cluster management and deploys tasks in Docker containers
        \end{itemize}
        \vspace{-10pt}
		{\bf $\rightarrow$ Software Renderer} \hfill \emph{Jul 2019} \\
        \vspace{-10pt}
        \begin{itemize}
        \item Developed a software-based rasterizer and renderer with pixel and vertex shader support in C++
        \item Capable of barycentric interpolation, backface culling and block-based rasterization
        \end{itemize}
		\vspace{-10pt}
		{\bf $\rightarrow$ LASSO/Wavelet Based Compressed Sensing} \hfill \emph{Jul 2019} \\
        \vspace{-10pt}
        \begin{itemize}
        \item Computes LASSO on the matrix-vector product representation of the discrete wavelet transform of an input signal with orthogonal Daubechies wavelets
        \item Can lossily compress audio/images to any amount or preprocess them for ML training purposes
        \end{itemize}
		\vspace{-10pt}
		{\bf $\rightarrow$ TaxiFindMe} \hfill \emph{Apr 2018} \\
        \vspace{-10pt}
        \begin{itemize}
        \item Routing web app that helps New Yorkers find the best spot to minimize taxi waiting time, taking into
account travel time and time of day
        \item Preprocessed 20 million entry taxi dataset with k-means machine learning algorithm; for querying, KNN is run from an input location to find nearest cluster. Frontend employs Django
        \item Reduced query time up to 94\% from the naive implementation
        \end{itemize}
		\vspace{-10pt}
		{\bf $\rightarrow$ ShirtMapper} \hfill \emph{Jan 2018} \\
        \vspace{-10pt}
        \begin{itemize}
            \item App that resizes images of custom shirts and maps them onto people
            \item Utilizes OpenCV and Scipy, and uses Haar classifiers for edge detection; frontend employs React Native
        \end{itemize}
		\vspace{-10pt}
        \section{\titleline{\centerline{SKILLS}}}
		\vspace{6pt}
		{\bf Awards:} Exploravision National Contest \hfill \emph{2016} \\
        \vspace{-10pt}
        \begin{itemize}
		    \item Wrote a paper proposing blockchain's potential link to autonomous vehicles, and won honorable mention.	
        \end{itemize}
        \vspace{-10pt}
        {\bf Models/Algorithms:} Regression/classification (ridge, logistic, SVM, decision trees, OLS), PCA/SVD, \\
        \vspace{-0pt}
        \hspace{100pt} ensemble learning, k-means, deep learning (CNNs, LSTMs, GANs), Frank-Wolfe \\
            {\bf Frameworks/Softwares:} Numpy, Scipy, Pytorch, OpenCV, Docker, Slurm, d3js, OpenCL, OpenGL \\
			{\bf Programming Languages:} Python, Java, C, C++, C\#, Bash, Latex, SQL, JavaScript, Matlab \\
			{\bf Operating Systems:} Unix-like systems (Linux, FreeBSD, Mac OS X), Windows \\
			{\bf Certifications:} Android Development (University of Maryland through Coursera) \\
		\vspace{-6pt}
			
	\end{resume} 
\end{document}
