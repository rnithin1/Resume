\documentclass{res} 

\usepackage{pifont}
\newsectionwidth{0pt}  
\usepackage{fancyhdr}
\usepackage{multicol}
\usepackage[bottom=1in,top=-0.2in,left=1in,right=1in]{geometry}
\renewcommand{\headrulewidth}{0pt}
\setlength{\headheight}{24pt}
\setlength{\headsep}{24pt}
\pagestyle{fancy}
\long\def\/*#1*/{}
\cfoot{}

\begin{document}
	\thispagestyle{empty} % no header
	\centerline{\bf \Huge{Nithin Raghavan}}
    \centerline{(678) 200-5839 | rnithin@berkeley.edu | rnithin1 (Github) | linkedin.com/in/nithinraghavan}
	\begin{resume}
	\vspace{-20pt}

\section{\centerline{EDUCATION}}
		\vspace{6pt} 
		{\bf University of California, Berkeley (Class of 2020)} \hfill \emph{Aug 2017 -- present} \\
		Computer Science {\sl Bachelor of Arts}, Applied Mathematics {\sl Bachelor of Arts} \hfill (\emph{GPA: 3.59}) \\
		\vspace{-20pt}
		\begin{multicols}{2}
		\begin{itemize}
		\item CS61B: Data Structures%{\it (Intended)}
		\item CS170: Efficient Algorithms
		\item EE127: Optimization Models and Applications
		\item Math 128a: Numerical Analysis
		\item Math 126: Partial Differential Equations
		\item CS189: Introduction to Machine Learning %{\it (Intended)}
		\end{itemize}
		\end{multicols}
		\vspace{-25pt}
		\vspace{6pt}
		{\bf Georgia Institute of Technology} \hfill \emph{Aug 2015 -- May 2017} \\
		Courses Taken in High School
		\vspace{-10pt}
		\begin{multicols}{2}
		\begin{itemize}
		\item Applied Combinatorics
		\item Number Theory and Cryptography 	
		\end{itemize}
		\end{multicols}
		\vspace{-15pt}
		\vspace{-2pt}
		\section{\centerline{EXPERIENCE}}
		{\bf Samsung SARC/ACL} \hfill \emph{May 2019 -- Present} \\
        \vspace{-10pt}
        \begin{itemize}
            \item Conducted extensive research on deep learning usecases and models as part of Samsung's GPU team
            \item Analyzed several machine learning models for routes of optimization and quantization
            \item Wrote and implemented OpenCL and OpenGL code 
            \item Researched the graphics pipeline and became acquainted with AMD's compute and graphics architecture
            \item Wrote, trained and implemented a neural network to perform ambient occlusion on complex scenes
            \item Currently writing a neural network to perform style transfer using InstaGAN from images rendered with Lambertian shading in OpenGL to images rendered with Physically Based Rendering 
        \end{itemize}
		\vspace{-10pt}
		{\bf Mobile Sensing Lab, UC Berkeley} \hfill \emph{Oct 2018 -- Present} \\
        \vspace{-10pt}
        \begin{itemize}
            \item Currently writing code implementing a parallelized Frank-Wolfe algorithm for dynamic traffic assignment in C++/CUDA using contraction
                hierarchies
            \item Helping research the impact of different optimization models of routing behaviour on the Waze problem
        \end{itemize}
		\vspace{-10pt}
		{\bf RISE Lab, UC Berkeley} \hfill \emph{Jun 2018 -- Dec 2018} \\
        \vspace{-10pt}
        \begin{itemize}
            \item Designed and implemented a data visualization tool for Jupyter Notebook for hyperparameter opti-
mization for Cirrus, a serverless machine learning framework
            \item Helped write code involving AWS Lambdas for model primitives such as logistic regression
        \end{itemize}
		\vspace{-10pt}
		{\bf IBM Almaden Research Center, Machine Learning Laboratory} \hfill \emph{Jul 2017 -- Aug 2017} \\
        \vspace{-10pt}
        \begin{itemize}
            \item Trained an artifical neural network with visual question answering abilities on Stanford’s CLEVR
dataset with 70\% overall accuracy
            \item Implemented sequence autoencoders, CNNs and LSTMs with Tensorflow and Keras
        \end{itemize}
		\vspace{-10pt}
		\section{\centerline{PROJECTS}}
		\vspace{6pt}
		{\bf Resource-Provisioning GPU Server} \hfill \emph{Dec 2017 -- present} \\
        \vspace{-10pt}
        \begin{itemize}
            \item Developed a Python-based shell to automate on-demand request processing and resource provisioning
in a GPU + CPU cluster
            \item Collaborated on a team to create a program that utilizes Slurm for cluster management and deploys tasks in Docker containers
        \end{itemize}
		\vspace{-10pt}
		{\bf TaxiFindMe} \hfill \emph{Apr 2018} \\
        \vspace{-10pt}
        \begin{itemize}
        \item Routing web app that helps New Yorkers find the best spot to minimize taxi waiting time, taking into
account travel time and time of day
        \item Preprocessed 20 million entry taxi dataset with k-means machine learning algorithm; for querying, KNN is run from an input location to find nearest cluster. Frontend employs Django
        \item Reduced query time up to 94\% from the naive implementation
        \end{itemize}
		\vspace{-10pt}
		{\bf ShirtMapper} \hfill \emph{Jan 2018} \\
        \vspace{-10pt}
        \begin{itemize}
            \item App that resizes images of custom shirts and maps them onto people
            \item Utilizes OpenCV and Scipy, and uses Haar classifiers for edge detection; frontend employs React Native
        \end{itemize}
		\vspace{-10pt}
		\section{\centerline{SKILLS}}
		\vspace{6pt}
		{\bf Awards:} Exploravision National Contest \hfill \emph{2016} \\
        \vspace{-10pt}
        \begin{itemize}
		    \item Wrote a paper proposing blockchain's potential link to autonomous vehicles, and won honorable mention.	
        \end{itemize}
        \vspace{-10pt}
        {\bf Models/Algorithms:} Regression/classification (ridge, logistic, SVM, decision trees, OLS), PCA/SVD, \\
        \vspace{-0pt}
        \hspace{100pt} ensemble learning, k-means, deep learning (CNNs, LSTMs), Frank-Wolfe \\
            {\bf Frameworks/Softwares:} Numpy, Scipy, Sk-learn, Pytorch, OpenCV, Docker, Slurm, d3js, CUDA \\
			{\bf Programming Languages:} Python, Java, C, C++, C\#, Bash, Latex, SQL, JavaScript, Matlab \\
			{\bf Operating Systems:} Unix-like systems (Linux, FreeBSD, Mac OS X), Windows \\
			{\bf Certifications:} Android Development (University of Maryland through Coursera) \\
		\vspace{-6pt}
			
	\end{resume} 
\end{document}
